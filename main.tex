\documentclass[12pt]{article}
\usepackage[utf8]{inputenc}
\usepackage[margin=1.2in]{geometry}
\usepackage{setspace}   %Allows double spacing with the \doublespacing command
\renewcommand{\familydefault}{\sfdefault}
\usepackage{helvet}

\title{AMC Entertainment Holdings Case Study}
\author{Adithya Shastry, Brian Long, and Merve Kahraman}
\date{April 2019}

\begin{document}

\maketitle
\newpage
\tableofcontents
\newpage
\doublespacing
\section{Industry History}


The American movie theatre industry has been around for over 100 years, beginning with the Nickelodeon  in June of 1905. The industry subsequently exploded among the American populous, reaching an industry height in the 1930’s during the great depression, where over 65\% of the entire U.S. population visited the movie theater weekly.\footnote{Pautz, Michelle. "The Decline in Average Weekly Cinema Attendance: 1930 -2000." Issues in Political
    Economy, vol. 11, 2002, pp. 1-18
} Numerous factors aided in the initial boom of the movie industry, a lack of substitutes, significantly lower ticket prices when compared to today, and historical context. In particular, many sociologists point to the great depression as a particularly significant force driving people into the theaters. That is, the depression acted as a sort of historical compliment to the movie theater industry as it allowed an escape from reality for the general population. A particular quote from a 1979 paper reads:

\\

"Throughout most of the Depression, Americans went assiduously, devotedly, almost compulsively, to the movies…the movies offered a chance to escape the cold, the heat, and loneliness; they brought strangers together, rubbing elbows in the dark of movie palaces and fleapits, sharing in the one social event available to everyone."\footnote{Pautz, Michelle. "The Decline in Average Weekly Cinema Attendance: 1930 -2000." Issues in Political
    Economy, vol. 11, 2002, pp. 1-18
}

\\

At the time, the movie theater industry was able to gain footing with audiences because of the experience they offered. Going to the movies was seen as a temporary escape from reality which was not offered anywhere else. By the end of the depression, the effect of this escape had begun to wear, and the percentage of weekly moviegoers accounted for just under 50\% of the U.S.\footnote{Pautz, Michelle. "The Decline in Average Weekly Cinema Attendance: 1930 -2000." Issues in Political
    Economy, vol. 11, 2002, pp. 1-18
} population. However, following the beginning of World War 2, movie theaters were able to find a second wind by showing newsreels, becoming most Americans’ only visual representation of the War effort, once again reaching around 62\% of the population.\footnote{Pautz, Michelle. "The Decline in Average Weekly Cinema Attendance: 1930 -2000." Issues in Political
    Economy, vol. 11, 2002, pp. 1-18
} Shortly after this boom, the movie industry suffered its first- arguably its largest-  hit to revenues with the widespread distribution of the television. Despite existing in some experimental capacity since the 1920’s, the television would not see widespread distribution until after World War 2. Suddenly, Americans had the ability to consume digital media in the comfort of their own home for a single fixed purchase. Between 1950 and 1955, the number of American households containing  television saw an 87\% increase, while the percentage of weekly moviegoers had dipped to around 25\%.\footnote{Pautz, Michelle. "The Decline in Average Weekly Cinema Attendance: 1930 -2000." Issues in Political
    Economy, vol. 11, 2002, pp. 1-18
} Following this, American movie Theaters saw the emergence of the multiplex in the 1960’s. These new, larger theaters contained multiple screens, and would show multiple movies simultaneously at varying times throughout the day. Interestingly, weekly movie theater attendance continued to drop, finally reaching a nadir at just under 10\%. The 1990’s saw the widespread adoption of the internet, which subsequently saw the birth of both online streaming and internet piracy. In 1999, the website known as Napster popularized the process of sharing digital medical online, regardless of copyright claims. Despite being dissolved in 2001, the “peer-to-peer” process of digitally copying and distributing media online which napster popularized would subsequently give birth to hundreds of copycat sites. Effectively, once movies were released on DVD, those who purchased them were able to obtain a digital copy from said DVD and subsequently distribute it online to numerous napster-esque sites. By 2008, it was estimated that these piracy sites had cost the movie industry around \$1.3 billion. By 2010, the number had skyrocketed to \$6.7 billion, and finally to \$31.8 billion in 2016. It is currently estimated that the figure will reach nearly \$52 billion by 2022.\footnote{Reporter, DTVE. "Piracy to cost TV and film industry US \$52 billion by 2022." DigitalTV.com, 30 Oct. 
     2017, www.digitaltveurope.com/2017/10/30/piracy-to-cost-tv-and-film-industry-us52bn-by-2022/. 
     Accessed 8 May 2019. } Additionally, the early 2000’s also saw the emergence of online streaming services such as Netflix, Hulu, Amazon Prime, etc. For a monthly fee, these services allowed members to view countless movies and television shows from their personal devices at any time in HD quality. 

\section{Introduction}


%WE SHOULD ADD MORE TO THIS SECTION!

%I HAVE COPIED THIS FROM THE PROPOSAL



AMC Entertainment Holdings, Inc. is involved in the theatrical exhibition business. It was founded in Missouri in 1920. It owns and operates theaters in the US and Europe and is a subsidiary of a Chinese private conglomerate, Dalian Wanda Group Co. AMC is the world’s largest theatrical exhibition company and it offers a broad range of entertainment such as traditional films, independent films, performing arts, music and sports. The revenues are generated primarily from ticket sales, theatre food and beverages. There is also revenue generated from ancillary sources such as advertising, AMC Customer Loyalty program, renting of theatre auditoriums and online ticketing fees.

The general value proposition which for the movie theater industry is the offering of newly released movies, unable to be viewed elsewhere, at a fixed ticket price. The major competitive advantage over their substitutes is the experience that a customer gets when they watch a movie in the theatre versus watching a movie at home or elsewhere. In general, the movie theatre market has seen a decline because of quality of the content on major streaming services like Hulu, Netflix, and Amazon Prime. These streaming services have captured significant market share from the movie theatre industry because of the convenience associated with these products from the consumer’s perspective, wide array of content, and affordable pricing. As a result, the remaining theatre brands have been competing to retain as much of the market share left in the industry.

In order to maintain its market share, AMC must compete with both Cinemark and Cineworld. Further, as we observed in the spotify case, many sources of paid digital media have a difficult time maintaining profit levels due to the abundance of substitutes available through the internet. The same rule holds for AMC as well as all other movie theaters in the industry. Many potential consumers are less likely to spend the money to go and see a new movie if they believe that it will soon be available on a streaming service which they are already paying for such as Amazon Prime or Netflix. As mentioned above, these online streaming services represent substitutes for AMC, and thus drive the demand for their product down. Similar to spotify, movie theatres must also ‘compete with free’, that is, most new movies are eventually posted illegally online in HD quality. Put simply, AMC must find a way to differentiate their product from the existing substitutes, therefore increasing their overall demand. Amazon, Netflix, and other online sources gain most of their business due to the convenience that they offer, whereas a movie theater requires that you physically travel and purchase a ticket for a specified time, streaming services allow you to watch a given movie on your at any time and any place. In order to combat this, many theaters agreed to offer a moviepass subscription service, which gave rewards and lower prices to frequent movie watchers. 


\section{External Market Analysis}
%Porter's Five forces


\subsection{Supplier Relations}
The modern movie industry oligopoly began after World War I with the rise of a collection of studios located in and around Hollywood. The locus of production and distribution decision making was still centered in southern California. The coming of solidified Hollywood’s control over the world cinema market, and moved the film industry into the studio era. Film making, distribution and exhibition were owned and dominated by five corporations: Paramount Pictures, Loew’s, Fox Film, Warner Bros and RKO. These corporations ruled the Hollywood and operated around the world as fully vertically integrated business enterprises. The Big Five owned the most important movie theaters in the US. By controlling picture palaces in all of US’ downtowns, they took in three quarters of the average box office take.




\subsection{Threats to entry}
The barriers for a new theater company to enter the market are dependant on a number of factors such as the upfront cost associated with opening a new movie theater, creating relationships with suppliers, finding a profitable location, and the marketing costs.The notion of creating relationships with suppliers is what is most difficult to do since there is very little leverage a new theater firm has over its suppliers and therefore would not be able to negotiate deals for the right to show movies effectively.Finding a profitable location is also a major consideration that new movie theater owners need to make before stepping into the industry. For example, the new location of the theater needs to include the following: enough consumers and an absence of direct competitors. Achieving both of these points will ensure that the new theater has a shot at being successful. Finally, the success of the marketing an up and coming theater does is integral to its overall success in the industry. Simple because when a new theater opens it needs to market its brand to potential customers otherwise it is very unlikely people will leave an already known brand for a new one. Given all of this evidence, it is quite unlikely that a new theater company will enter the market in any capacity without facing serious push back from the way the industry is set up.
\subsection{Complements}
The main complements that AMC has are its food items and beverages. Improvements in these services will draw in more customers to their theaters. According to their Annual report, food and beverages account for a lot of the overall revenue of the company itself.
\subsection{Substitutes}
As mentioned in the industry history section, many of the industry's financial problems arise from the prevalence of substitute goods. In the early 50's the greatest threat was the newly distributed television. In more modern times, the most notable substitute goods are internet piracy and online streaming services. Netlfix, Hulu, and Amazon Prime all allow subscribed customers to watch an impressive multitude of movies and television shows on demand and in HD quality for a monthly fee. While the services may not have all films available in theaters, they nonetheless offer the convenience of being able to view films and television any number of times from any device with the website available at any time of day. Further, internet piracy similarly allows users to view virtually any movie or television show ever created online and in high quality. Much like streaming services, internet pirates can view these forms of media from any device and at any time of day, so long as they have access to WiFi. More importantly, while Movie theaters offer film viewings at price per-view and the streaming services offered their media for a monthly fee, internet pirates can access these films and shows for no fee whatsoever. Finally, movie consumers will always have the option to simply wait for a film to be released on DVD. For a fixed fee, consumers will be able to possess a hard copy of any film, and subsequently be able to view it any number of times at any time of the day so long as they have a device to play them. Finally, as was the case in the 50s, Cinemas must compete with television. Today, there are channels which constantly play movies at every hour of the day. While these televisions do not show movies at release, more price sensitive customers may simply opt to wait for a given movie to be shown on television rather than choosing to watch it at release. 
\subsection{Buyer Relations}
AMC is currently competing with many substitutes for the same market share and has a product that currently has a very elastic demand curve, therefore it is clear that the overall leverage, for AMC, over its customers is very minimal.
\subsection{Competitors}
Here the direct competitors of AMC theaters will be analyzed with a brief description of their business and a quick overview of their strengths and weaknesses as can be gathered from various resources. 

\subsubsection{Regal Cinema Companies}

During the 1990s, the Knoxville, Tennessee-based company grew from a single twin theater in Titusville, Florida to the world’s largest theater chain. There were 3,672 screens in 406 locations in 30 states. In 1994 the company went public and the CEO Michael L. Campbell acquired National Theaters, then the Litchfield chain and Neighborhood Entertainment, Georgia State Theaters and the Cobb circuit.  By 1998, annual revenues totaled more than \$700 million with further plans of expansion. Forbes estimated that in 1997 its profit margin was 14\% with \$320 million in sales with a market value of \$1 billion. Regal was able to create profits from small town cinemas with skilled management. 

\subsubsection{Carmike} 

In 1982 Carl Partick bought the Martin Theater chain of Georgia and renamed it after his two Sons, Carl Jr and Mike. Carmike’s strategy was to monopolize exhibition in small towns and medium sized cities across the US. Initially the company operated only in the South. In 1999, it had more than 500 theater complexes with 2837 screens. It ranked second with operations in 36 states all through the South and Middle America. Carmike operated in media markets with less than 200,000 people to avoid competition with Loews Cineplex, National Amusements and AMC. In 1997, Carmike formed a joint venture with Wal-Mart to develop family fun centers to be called “Hollywood Connections”. 

\subsubsection{AMC Entertainment Inc }

Based in Kansas City, Missouri, AMC’s history stretched back to the pioneering days of movies in the mall. Originating from the small independent Durwood chain in the 1920s, AMC expanded with drive-ins in the 1950s. In 1963, it opened the first twinned screen cinema, the Parkway One, with 400 seats with a single ticket booth and a concession stand. The Parkway cost \$400,000. As AMC’s profits exceeded  their expectations, they opened a six-plex by 1969. Throughout the 70s, they added more screens. In 1983 AMC went public owning over 700 screens around the Midwest. In 1995, AMC Grand Cinema complex opened in Dallas Texas. Within two years, this complex attracted 3 million patrons per year. The Durwood family bult 60 new mega complexes at a cost of \$30 million each. Their revenues averaged 10\% more than the industry. AMC deliberately clustered its cinemas in major markets against Loews Cineplex in Los Angeles, Houston, Dallas, Denver, St  Louis, Kansas City, Tampa, and Washington DC. AMC had a ‘frequent guest’ brand loyalty program which led to credit card offerings with regular attendance. In April 1999, AMC became the first theater to charge in excess of \$8.00 for admission to regular shows in Los Angeles. It also raised its top ticket price in Santa Monica, Century City and Woodland Hills. 

\subsubsection{CG Theaters}

In 1999, General Cinemas, a Newton Massachusetts based company, was the sixth largest chain in industry. It has been the leader in the ‘upscale’ theater experience. Prices of \$12 to \$15 for this “Premium Theater” was a record high. It offered amenities as special seating, limited access and first class restaurant. 

\subsubsection{United Artists Theater Circuit }

Based in Denver, UA followed AMC in multiplexing in the 1960s. However, it lost \$87 million in 1995, \$67.5 million in 1996 and \$13 million in the first half of 1997. In 1990s, it had no prospect of making money. Early in 1999, with more than 2000 screens worldwide UA remained a presence in the US and Latin America. 

\subsubsection{Hoyts Cinemas}

Based in Australia, Hoyts owned and operated 1542 screens worldwide with 945 in the US. As Regal and Carmike were expanding in the South, Hoyts took charge of the Northeast. It was also building cineplexes in Argentina, Austria, Chile, Germany, Mexico and New Zealand. 

\subsubsection{Others} 

Although these companies owned about three quarters of all screens in the US, there were smaller chains to seek niche segments. In 1999, Edwards Theater Circuits Inc. based in Newport Beach, California had 775 screens in California. Edwards added new complexes with larger leg room, stadium style seating, larger screens, fresh pizza and digital sound systems. Although it was not one of the large chains in terms of number of screens or revenues, it had an above average profitability. Based in Madison, New Jersey, The Clearview Cinema Group sought another niche. It targeted affluent senior citizens and parents with small children who did not approve of violence or sex in the movies. It then expanded into New England and Middle Atlantic States. Another niche driven chain was founded by former basketball player Magic Johnson. In 1994, Johnson started Magic Johnson Theaters in predominantly black neighborhoods in a partnership with Sony. 




\section{Internal Analysis}

This section will provide an in depth analysis of the overall business model in terms of value creation. This analysis will look at the AMC business model under the lens of Micheal Porter's value chain model. This model includes the following elements: Inbound logistics, Operations, Outbound logistics, marketing and sales, and services.Each of these chains will be described and subsequently analyzed as they are applicable to the business model of AMC theaters. Each of these aspects of the value chain will be analysed using a VRIO analysis. VRIO takes into account the activity's value, rarity, imitability, and finally the extent to which the organization can exploit this particular feature. 
\subsection{Inbound Logistics}
The inbound logistics of AMC theaters consists of how the company deals with selling tickets and getting customers to the theater. In the past, this was primarily done by selling tickets to customers at the counter before they were allowed to go to the movie. Nowadays, much of this process is done online using both their own platforms and also other platforms such as Atom Tickets, Fandango, and Movietickets.com. These online options for tickets have significantly improved the traditional process of buying tickets because of the convenience it offered customers. 

AMC also claims to use seat reservations in order to improve their customer's ability to sit in their personal ideal seat in the theater. In the past, the idea of reserving seats was nonexistent, and usually resorted to a first come first serve basis. This new addition adds some more convenience for the customers of AMC theaters because it gives them the ability to make decisions on buying tickets with the current state of the seating situation in mind. 

Looking at these advancements in their inbound logistics activities, it seems like AMC is doing the right thing since their moves are providing a better experience for their customers.Thus it is pretty clear that their operations in this regard are very valuable. However, it is very difficult to say that these moves are rare or imitable. This is because it is clear that these practices are done by most of their competitors to various extents. Overall, it is clear that this operations are at the level necessary to compete with other competitors, but not innovative enough to warrant a classification as a competitive advantage.  

\subsection{Operations}
This activity includes many activities that the company does in order to generate more revenue as the customer is in the movie theater itself. This includes the sale of concessions, sale of merchandise, and the quality of a customer's experience. In terms of the sale of concessions, the company has many many strides such as including various options other than just the traditional movie theater concessions such as popcorn, soda, and nachos. They have begun to also sell full meals to their customers through their dine-in movie initiative. When customers choose this option, they are given a full menu of a variety of food from which they can order food. This essentially allows AMC to become a one-stop-shop for costumers on their night of fun. For example, it is pretty common for people to go to dinner before going to a movie, but with this new initiative customers would be able to enjoy their dinner while at the movies.Next, we have the sale of alcoholic beverages at the theater. This, again, will serve as another form of revenue for the theater. The theater continues to sell movie specific merchandise at its concession stands in order to generate more revenue. No moves have been done in order to improve this revenue stream for the company. Finally, we have the improvements in newer technology. For example, AMC boasts IMAX, DOLBY Cinema, and Prime. All of these technologies add more options for customers who want these experiences when they go to the movies. 

It is clear that from these operations are indeed very valuable to the company because of their nature and the amount of money they generate for the company as a whole. However these very same practices are implemented by most of AMC's competitors. For example, Studio Movie Grill (SMG) offers dine-in options at all of their movie theaters. Since this is their focus, it is very difficult for AMC theaters to compete with them because of the learning advantage that SMG already has. In terms of the technology being implemented at AMC theaters. Since these technologies are not internally developed and therefore not proprietary, it is very easy for other competitors to implement these very technologies in their own theaters. In fact, many have already done this including major a major competitor like Cinemark. Through this analysis it is very difficult to classify the operations of AMC theaters as a competitive advantage over its competitors. 

\subsection{Outbound Logistics and Service}
The outbound logistics and service activities, under the application of their respective definitions of Micheal Porter, do not really apply to the overall value proposition of AMC movie theaters. 

\subsection{Sales and Marketing}
%This shows a clear competitive advantage: AMC has the best subscription service out of the bunch. They simply need to have a good turnover rate in terms of customers buying concessions and other things

Sales and Marketing are very important parts of AMC's business model. Since, as described earlier, movie tickets are essentially a commodity in the eyes of the consumer. This is because the experience that a customer gets when they attend any movie theater is essentially the same. In other words, the movie watching experience has so little differentiation among the competitors in the market. In fact the only differentiation that a customer faces when deciding on a movie theater, is the proximity to his or her location. Given this knowledge it is an integral part of the business model of many of these theaters to employ loyalty programs and other schemes in order to keep customers from going to other theaters. Some examples of these programs can be seen in AMC's stubs Alist program. The stubs Alist program is a loyalty program based around a subscription fee that a customer pays for various benefits such as three free movies a week, reduced rates on concessions, and waived online fees. This is a significant benefit for the company since it will give the company recurring revenue through concessions and other supporting activities associated with operations primary activity. AMC also employs various types of theaters such as classic theaters, Dine-In theaters(as described above), and normal theaters. The classic theaters offer a new experience for the customer in the form of a more intimate environment since these theaters are significantly smaller than regular theaters. 

As described above, it is very clear that the improvements mentioned above are indeed very valuable for the company. For example, the Stubs A list program has already increased attendance for theaters by 36\% because of the program\footnote{AMC Entertainment Holdings. (2018). Annual report 2018. Retrieved from SEC EDGAR Online database.}. However, a similar program like this can be imitated by other competitors. For example, Cinemark has its own rewards program called Cinemark Movie Club. However, this program gives very little benefits in comparison to AMC's stubs A list. Therefore, it shows that AMC has a cost structure that is better than the cost structure that Cinemark has and therefore has a rare and imitable practice. Given, the increase in attendance for AMC above, it is clear that the company can an will exploit this new advancement. Given the above information, it is clear that the company has a clear competitive advantage over its competitors in terms of their marketing strategy. Costumers acquired through these marketing methods simply need to be converted to customers that will buy concessions.

\section{Dynamic Change Analysis}

%Look at the effect that tech development has in the industry
From the above analysis, it is clear that technology is a very major part of the overall development of AMC as a whole in the theater industry. Developments in technology include scientific advancements in terms of the way movies are presented to consumers and by the different tactics used by AMC to get more costumers into their theaters. The later can be seen in the way the company has developed better inbound logistics. For example, the stubs Alist program has significantly increased AMC's profits by taking customers who had initially signed up with Movie Pass before they failed. Another example is the ease of which customers can buy tickets online or on a mobile device through the use of a variety of platforms. These developments in business practices reduces a large amount of the hassle consumers have when buying tickets to the movies. The more scientific advancements include ones that improve the sound quality, video quality, and other improvements that make the experience for the customer better in general. For example, AMC has improved their theater technology though the implementation of Dolby digital, IMAX, and other technological advancements. This is meant to give customers the best movie-going experience possible as well as a number of options based on each and every customer's specific preferences.






\section{Current Positioning in the Market}
In order to properly understand where AMC falls in the market as a whole, it is important to understand a number of factors, namely their strengths, weaknesses, opportunities, and threats. As shown above, AMC is by far the market leader in the North American movie theatre industry. Further, as we will mention below, their greatest competition comes from their substitutes rather than their competition. 

\subsection{Strengths} 
Based on the analysis of the internal processes of AMC theaters it is clear that the company has a few major advantages, or strengths, over their competitors. Some of the major strengths include their marketing activities. A clear example of this can be seen in the way the company successfully implemented their subscriber rewards program: AMC Stubs Alist. It is also clear that large majority of the success AMC is seeing right now is based around there superior marketing strategies. For example, even though there are direct competitors in the market that have similar operating activities, AMC consistently performs better with respect to overall revenues which is clearly tied to their overall superiority of their marketing strategy.

\subsection{Weaknesses} 

Given the commodification of the movie theater industry, it is very integral for AMC to keep a steady lead in front of their competitors both from an operations perspective and from a marketing perspective if they hope to keep the market share they already have or to increase it. This is a serious weakness because of the advent of On-Demand Streaming services like Netflix, Hulu, Amazon Prime, and others. Since AMC theaters have lost their convenience factor to them, in contrast to the past, AMC must now try to out compete the this convenience with an experience customers cannot get anywhere else.


\subsection{Opportunities}
Some opportunities include pursuing more of a share growth strategy centered on revenue. This is where AMC would continue to differentiate its product and market it to essentially make the demand more inelastic. This would allow the company to attract more customers and increase their market share in the industry and there by generate more revenue. This is the best approach for AMC because of the overall trend in the industry as a whole as it moves to the more convenient online streaming services. 


\subsection{Threats}

AMC faces intense competition from others in the industry. There are also local theaters with a niche customer base which attract more loyalty. These often offer more intimate experiences which cannot be easily replicated on a national level by theaters like AMC. There is an immense decrease in the theater attendance overall  as a result of emergence of substitutes such as on demand streaming channels. The advent of the internet and introduction of alternative entertainment has been disruptive in the overall business model. Finally, we have the advent of online streaming services that are taking away market share because of the overall convenience that these services offer. Given this the overall strategy of the company must change in order to keep up with the current trends in the industry as stated in the strengths and oppertunities section.


\section{Strategy Proposal}

As shown above, the movie theater industry has consistently faced threats from numerous substitutes throughout its history. The first notable substitute found its way into the market in the form of the television, which, despite not offering movies with the same consistency or reliability as movie theaters, provided viewers with the convenience of visual digital media from the comfort of their own homes at a single, fixed price. Put simply, the television represented a significant low-end disruption for the movie industry, offering limited services with higher convenience at a fixed price. In order to combat this, many movie theater chains began to replace their theaters with multiplexes, larger, multi-screen locations which would give individuals more flexibility as to when they could view certain movies. However, this increased freedom did not have the desired result. The multiplexes demanded massive overhead to maintain, and as a result theaters drastically increased their ticket prices. Most consumers did not believe that the increased convenience of multiple movie showings was worth the increase, especially with the television now in full swing, and thus movie foot traffic continued to drop. Effectively, movie theaters attempted to try and match the convenience of televised movies by increasing the availability of a given film. However, this increased convenience made it more difficult for cinemas to fill a given theater at a given viewing due to the increased number of showings. That is, cinemas were making less money per movie screening. 
The emergence of online streaming services represented another low-end disruption. Despite not offering movies at the time of their cinema release, streaming services offered an expansive multitude of movies at all times in HD quality from any device owned by the member. Much like television, these services represented a low-end disruption for the movie industry. For a monthly fee, the services allowed individuals to view the same media available in theaters at a later date and on a smaller screen. 
Further, the widespread use of online internet piracy sites further detracted from the theater industry. As was the case with Spotify, movie theaters as well as streaming services are always “competing with free”. That is, consumers will always have the option to illegally view virtually any movie or television show online in HD quality from a computer. Unsurprisingly, this convenience has detracted heavily from the movie industry, reaching a shocking \$31.8 billion in 2016. \footnote{Reporter, DTVE. "Piracy to cost TV and film industry US \$52 billion by 2022." DigitalTV.com, 30 Oct. 
     2017, www.digitaltveurope.com/2017/10/30/piracy-to-cost-tv-and-film-industry-us52bn-by-2022/. 
     Accessed 8 May 2019. }
Generally speaking, the movie theater industry has consistently been plagued by more convenient, easy to use options throughout its history. Therefore, their best option to survive is through product differentiation. Even as early as the 1930’s, people cited the ‘experience’ of going to the movies as the industry’s main reason for success.\footnote{Pautz, Michelle. "The Decline in Average Weekly Cinema Attendance: 1930 -2000." Issues in Political
    Economy, vol. 11, 2002, pp. 1-18} In order to maintain profitability AMC’s best bet is to recapture the uniqueness of a trip to the movie theater. AMC has already invested significant funds into exacting this goal, rennovating their theaters with luxury seating, improving the technological abilities of their theaters with their Dolby Theaters initiative, offering a repeated customer benfit service called AMC stubbs, and offering a full dine in service with restaurant quality food.\footnote{"AMC History." AMC Movie Theaters, 8 may 2019, www.amctheatres.com/corporate/amc-history. Accessed 8 
     May 2019. }


These efforts have already helped AMC significantly. From 2012 to 2018, their revenues have spiked, from \$811 million to \$5.4 billion, showing a 666\% increase over the last 6 years.\footnote{AMC Theatres. "Amc Theatres' Revenue from 2006 to 2018 (in Million U.S. Dollars)." Statista - The Statistics Portal, Statista, www.statista.com/statistics/206959/revenue-of-amc-theatres/, Accessed 7 May 2019} Put simply, AMC has focused their efforts into re-defining what it means to take a trip to the movies. The emergence of dine-in services, AMC Stubs, renovated theaters, and vastly improved technology has allowed AMC to differentiate their product line from their substitutes. Effectively, they are fighting the convenience of streaming and piracy services by offering a bundled experience of comfort seating, restaurant-quality food offerings, and a technologically superior viewing experience. Going forward, AMC's best options are to continue this offensive differentiation strategy. They will never be able to offer the convencience of streaming services, nor the virtually non-existent price levels of internet piracy. Thus, their best option is to further improve the moviegoing experience and maintain their control over the higher end of the market. As mentioned in their 10k report, the emergence of dine-in services has increased the percentage of moviegoers who purchase food items from around 64\% to over 71\%.\footnote{AMC Entertainment Holdings. (2018). Annual report 2018. Retrieved from SEC EDGAR Online database.} That is, it is likely that at least 7\% of moviegoers who were not initially purchasing food items now purchase dine-in services. Of course, this figure does not necessarily represent the total number of patrons purchasing dine-in services as many customers who previously bought smaller food items may have switched to the full dine-in option. In order to further increase profits, AMC should focus on enticing more customers to use their dine-in option. In order to facilitate this, AMC could offer a rewards service which offers bonuses both for frequent movie watchers as well as frequent dine-in users. Dunkin Donuts has recently been able to generate massive successes using their DDperks option, both rewarding individuals for purchasing Dunkin items as well as enticing them to return. AMC could expand their AMC stubbs service to reward individuals for watching multiple films, while offering more substantial rewards to those who also purchase dine-in service. Thus, AMC would both entice more individuals to use their AMC Stubbs service, theoretically increasing their number of repeat customers, while also enticing existing AMC Stubbs users to purchase more dine-in services. 
Additionally, AMC has recently been able to drastically increase their revenues by implementing more iMAX locations as well as improving their sound systems with the emergence of Dolby Cinema.\footnote{"AMC History." AMC Movie Theaters, 8 may 2019, www.amctheatres.com/corporate/amc-history. Accessed 8 
     May 2019. } By improving the technical quality of the movie viewing, AMC was able to further differentiate itself from its substitutes That is, the technical resolution and sound quality of netflix streaming and internet piracy are vastly inferior to the options offered by AMC. AMC could further press this advantage by investing capital into the technical R+D behind movie viewing. Any new services or options they provide only serve to differentiate their product further from the low end of the market, increasing the quality of their product and therefore enticing more customers to use their service over netflix and online streaming. 
On top of all this, AMC needs to fundamentally change the way in which the general public views the experience of going to the movies. Their goal has to be occupancy of the top of the market, therefore their image cannot revolve around just the action of watching a movie alone. AMC should invest heavily in a new marketing campaign in order to reinvent their brand with a heavy focus on the overall experience of going to the movies. If customers view AMC just as a place to view films and nothing else, the wide majority of people will opt for the lower-end options. That is, the cost of waiting to see a movie online or on netflix is far less than the ticket price to watch the same movie. However, by stressing the movie viewing as well as the dine in services, the vastly improved amenities, technical quality, etc. AMC can differentiate themselves from their low-end competition, effectively justifying the price of admission.  
All the strategies mentioned above are focused on capturing customers who are interested in the higher end of the moviegoing market. Thus, while their focus is more on capturing new customers rather than protecting current share, the aim is not to access customers on the low end. If AMC wished to further increase their customer base, they could also focus their efforts on capturing some of the low-end customers using netflix and internet piracy. Many customers who use netflix and online streaming over movie theaters are more price sensitive than others, and thus would likely be susceptible to any pricing actions taken by AMC. If AMC wished to capture some of these customers, they could use more dynamic pricing with their ticket pricing for larger movies. That is, for movies that earn high box office numbers, they could offer discounted prices the longer the movie is out. Therefore, they would be more likely to capture the customers waiting until the movie is available online by offering a discounted price for waiting to watch until after the initial release. 


%Propose the strategy the company can use



\section{Strategy Implementation}
Their first action should be to link their AMC Stubbs service to the dine in-service. They could implement this by offering rewards to frequent moviegoers including discounted meals. They could further entice Stubbs users by offering special deals that benefit those moviegoers who frequently use the dine in option, perhaps tied to both discounted ticket prices and free food items. Secondly, they should invest funds into the technical side of their theaters. By staying on top of all new technologies regarding sound and picture quality, AMC can further widen the gap between their product and its substitutes. They could facilitate this research by partnering with IMAX Corp and Dolby Laboratories. This would give them strong leverage with their technological suppliers, which should theoretically give them an advantage over their competitors in the market. In the background of these efforts, AMC should invest heavily in a marketing campaign to re-brand themselves in order to stress the movie 'experience' rather than simply watching a movie. Commercials and advertisements stressing their high quality technology and full dining services in addition to the movie itself would aid in creating this new image. Finally, if AMC wished to access some of the lower-end customers, they could also offer dynamic ticket prices, which would offer discounted rates for especially successful movies a few weeks after release. In order to facilitate this, AMC could design an app, much like those used for airline tickets such as googleflights, which would notify individuals regarding the discounted ticket rates. This would additionally entice customers to see a given film more than once. In conclusion, AMC's biggest issues come from its substitutes in the industry. They will never be able to compete with the convenience or non-existent prices of streaming services and pircacy respectively. Their best option, therefore, is to differentiate themselves to justify the price of admission. That is, their best course of action is to convince the average person that the superiority of their service makes the fixed price per movie a worthwhile investment. 

%Show how the company will implement the strategy proposed



\section{Conclusion}
AMC holdings exists in an industry that has faced tremendous adversity externally from technological progress. Dating as far back as 1920 with the invention of the television streching all the way to the modern day, where internet piracy is expected to have stolen \$52 billion in revenues from the movie industry by 2022. As shown in the industry section, AMC lies at the top of the movie theater market, owning the most locations and generating the most revenues. As with any other business, AMC should always be conscious of their competition in the market, but their most pressing issue to address is the low end disruptions caused by online streaming and 'peer-to-peer' piracy.Their efforts with theater renovations, AMC Stubs, Dolby Theaters, and dine-in services have already aided in differentiating the theater chain from the low-end substitutes. In order to survive and subsequently prosper in the market, AMC must continue to press this growing advantage, tying their rewards service to include their dine-in services, remaining on tip of theater technology, and working to re-brand themselves. Further, by offering dynamic ticket pricing to more price sensitive consumers, AMC could attempt to access the low-end customers that they previously lost to stteaming and piracy. Additionally, all of these options will differentiate AMC from is substitutes as well as its competition. If AMC becomes the only theater chain to offer these high end services, they will effectively be differentiating themselves from their industry competition, allowing them to sell tickets at higher prices without losing any of their customer base, therefore capturing more margin. Put simply, these efforts stand to benefit AMC in two major ways. First, they will be pressing themselves further into the high end of the market, differentiating themselves from their substitutes and thus potentially capturing more customers previously lost to piracy and streaming. Second, they will be differentiating their product line in reference to their competition within the market, therefore allowing them to charge higher ticket prices with no loss of share. That is, if these strategies prove successful, AMC stands to both gain more customers at a higher margin. 



\end{document}
